\chapter{Building the application}

For a build system we chose GNU Autotools. This allows building
and installing the application with only three well-known commands:

\lstset{language=bash,caption=Build System,label=lst:building}
\begin{lstlisting}
  make -f Makefil.lkl
  ./configure
  make
  make install
\end{lstlisting}

There are few required packages:
\begin{itemize}
  \item flex
  \item libgtk2
  \item libgtksourceview
  \item libreadline
\end{itemize}

This can be installed, in a Debian based distribution, using the following 
command:
\lstset{language=zsh,caption=Required packages,label=lst:bpacakges}
\begin{lstlisting}
  $> sudo apt-get install libgtk2.0-dev libgtksourceview-dev flex autotools-dev libreadline-dev
\end{lstlisting}

The first step is used to build \textbf{LKL} from the Linux kernel sources; it
requires the \textbf{LKL} sources, their location is specified in a variable in
Makefile.lkl. This step is only required only on the first build or
when something has changed in the \textbf{LKL} sources.
