\chapter{Conclusions}
\label{chapter:conclusions}

Because of \textbf{LKL} is implemented as a virtual architecture, is is expected
that its performances are reduced when compared with a computer running \textb{Linux}.
There might be a speed bottleneck with the hub or bridge, but test have not proven or disproven
this. Although transmission speeds are not very high, the application is still very useful. Users can
see how protocols and devices work.

All of the main objectives have been implemented in the application:
\begin{itemize}
	\item All the devices work as expected, as it can be seen in the devices tests
	\item The topologies can contain a large number of devices, as proven in the scalability tests in \ref{sec:scalability}
	\item External applications can be connected to the \textbf{\project} devices, as it has
been done in \ref{sec:complex-network}
\end{itemize}

Using the GUI built over the hypervisor new users can create complex topologies and run simulations
on them more easily than it would have been using the command line tools and configuration files. Users will be able to configure
most of the aspects of the devices using this interface so there is no need to learn to use the command line. More than
this, for the devices the autocomplete feature of the readline library makes is easy to see available commands and
to run them.
