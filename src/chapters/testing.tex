\chapter{Testing}
\label{chapter:testing}

\section{Round trip time and throughput}
\label{sec:response-time}


We would like to test the response time and the throughput of one device. For this we will be using
a simple topology described in \figref{img:trans-test}. It contains a router with one interface and
a hub connected to it. The hub is added so we can connect a bridge to the router to be able to use
existing user space tools for testing.

\fig[scale=0.5]{src/img/trans.png}{img:trans-test}{Response time test topology}

The configuration of the devices:

\begin{center}
  \begin{table}[htb]
  \begin{center}
  \begin{tabular}{| l | l | l |}
    \hline
      Hostname & Interface & IP  \\ \hline
      Router0 & eth0 & 192.168.2.1/24 \\ 
    \hline
  \end{tabular}
  \end{center}
  \caption{Device configuration}
  \label{table:rtt-conf}
  \end{table}
\end{center}

After we start the bridge, we must bring up the interface and add a route for the network of the
router. This is done with the following commands:

\lstset{language=zsh,caption=Configuring a TAP interface,label=lst:conf-tap}
\begin{lstlisting}
  #> ip link set tap0 up
  #> ip address add 192.168.2.100 dev tap0 ;; Must use an address from the same network as the interface
of the driver
  #> ip route add 192.168.2.0/24 dev tap0
\end{lstlisting}

We will be using ping to test the devices. We chose ping because it was easier than writing another test application
that times the delivery of packets. The output of the command is:
\lstset{language=zsh,caption=,label=lst:pingout}
\begin{lstlisting}
 #> ping 192.168.2.1 -f -c 1000 -s 1400 -a -v
PING 192.168.2.1 (192.168.2.1) 1400(1428) bytes of data.
                 
--- 192.168.2.1 ping statistics ---
1000 packets transmitted, 1000 received, 0% packet loss, time 13242ms
rtt min/avg/max/mdev = 39.966/116.117/224.098/29.775 ms, pipe 17, ipg/ewma 13.255/106.991 ms
 #> ping 192.168.2.1 -f -c 1000 -s 1400 -a -v
PING 192.168.2.1 (192.168.2.1) 1400(1428) bytes of data.
                 
--- 192.168.2.1 ping statistics ---
1000 packets transmitted, 1000 received, 0% packet loss, time 13015ms
rtt min/avg/max/mdev = 9.414/114.034/203.237/32.603 ms, pipe 17, ipg/ewma 13.028/111.240 ms
\end{lstlisting}

\begin{itemize}
  \item \textbf{Throughput}. Throughput or network throughput is the average rate of successful message delivery 
over a communication channel. As it can be seen in the output, 1000 packets have been sent in 13242ms. As this
is a round trip, the throughput of the device is situated somewhere around 1000 packets per 6621ms. The cause of this
low throughput will be discussed at Round Trip Time.
  \item \textbf{Round Trip Time.}  Round trip time is the length of time it takes for a signal to be sent plus the 
length of time it takes for an acknowledgement of that signal to be received. Ping reports that rtt is
rtt min/avg/max/mdev = 9.414/114.034/203.237/32.603 ms. The network driver waits in a blocking state in a recv
call for the packets to arrive. This means that its speed is dependent on the host operating system. If it will
always be awaken fast when a packet arrives then the speed will be high, otherwise it will be low. This has been
noticed in practice with packets sent by ping having a rtt as low as 0.3 ms or as high as a 230ms.
  \item \textbf{Interpacket gap/Exponentially weighted moving average (ipg/ewma).} Interpacket gap is the minimum idle period
between transmission of Ethernet frames, this allows devices to prepare for reception of the next frame. Exponentially weighted moving
average applies weighting factors that decrease exponentially. This gives greater importance to recent data while still not discarding
old data. The values calculated by ping are ipg/ewma 13.028/111.240 ms.
\end{itemize}

We tested with a few different sizes for the payload of the packets and we did not see any important changes in these
factors. The only parameter that changes was the minimum round trip time.
This means that the bottleneck in the network is not the driver or the way it handles data, but the time
that the native system allows the driver thread to run.

Iotop reported that the I/O activity of the router stayed in the 20-40 kb/s zone, but we noticed a peek
at a 170Kb/s. This might have been an anomaly.

\section{Scalability}
\label{sec:scalability}

One of the objectives of the project was to be able to simulate a large number of devices. For this we have
created a topology generator written in python that can be found in scripts/generator.py. It generates a topology of
x by y switches connected in a full mesh. To create a 4 by 5 topology in the folder test the following command
should be issued:
\lstset{language=text, caption=Generating a topology}
\begin{lstlisting}
python scripts/generator.py 4 5 test
\end{lstlisting}

Ping between the corners of the topology:

\lstset{language=text, caption=Ping}
\begin{lstlisting}
PING 192.168.20.1 (192.168.20.1) 56(84) bytes of data.
64 bytes from 192.168.20.1: icmp_seq=1 ttl=64 time=686 ms
64 bytes from 192.168.20.1: icmp_seq=2 ttl=64 time=39.3 ms
64 bytes from 192.168.20.1: icmp_seq=3 ttl=64 time=1.27 ms
64 bytes from 192.168.20.1: icmp_seq=4 ttl=64 time=573 ms
64 bytes from 192.168.20.1: icmp_seq=5 ttl=64 time=202 ms
64 bytes from 192.168.20.1: icmp_seq=6 ttl=64 time=37.8 ms
\end{lstlisting}
It can be seen that the time is greater than the single device
tests, but this is normal as there are more devices in the topology(the shortest path in the topology crosses 7 devices).

Ping between the corners of the topology with the flood option:
\lstset{language=text, caption=Ping}
\begin{lstlisting}
PING 192.168.20.1 (192.168.20.1) 56(84) bytes of data.
.............................................^C                                
--- 192.168.20.1 ping statistics ---
5826 packets transmitted, 5781 received, 0% packet loss, time 75522ms
rtt min/avg/max/mdev = 100.066/568.539/1026.571/116.873 ms, pipe 79, ipg/ewma 12.965/631.160 ms
\end{lstlisting}

We have looked at the devices using top. The switches take between 1.6 and 2.1 Mb of memory. This is
because the switches have different number of interfaces. The ones in the corner having only 2 while
the ones in the middle have 4. Although the devices are given 16Mb of memory, the native system 
reports only the memory that is in use. CPU time was hard the notice as the devices are only active
when a packets passes their interfaces. Only when the flood option was used with ping, we noticed
4\% use of the CPU by a switch, but this also varied greatly. 

We have done further test on the firewall to see how much memory it requires when a large number
of rules are added. The required memory increases linear with the number of rules; every additional
100 rules require about 0.2MB of memory. This can be seen \figref{img:memory-test}

Another thing that we have noticed is the number of context switches that the native operating
system has to do when the device was flooded. Under normal circumstance these stayed in the hundreds
zone( 300-700 context switches per second) but when ping was issued, these rapidly increased to over
5000 context switches per second.

\fig[scale=0.5]{src/img/memory.png}{img:memory-test}{Memory use on firewall}

\section{Complex network}
\label{sec:complex-network}

The next test is more complex then the ones before, combining more features of the application into
one topology. We will be connecting three external applications to our \textbf{\project} topology. One of
them is a web server, serving request for the www.lklnet.org domain, and the other is a DNS server to
translate names for the lklnet.org zone. The third application is the browser with which we will
be accessing the web page. The \textbf{\project} topology contains one router and one switch and the hubs
that interconnect them. Bridges are then connected to these hubs to connect the external applications and
routes are added through the tap interfaces. The host computer should then be made to use the created
DNS server through the \textbf{\project} network. This can be done by adding nameserver 192.168.100.200
in /etc/resolv.conf. After this the web browser can be used to receive the web page. The topology is shown
in \figref{img:complex-test}

\begin{center}
  \begin{table}[htb]
  \begin{center}
  \begin{tabular}{| l | l | l | l | l |}
    \hline
      Hostname & Interface & IP & Interface & IP \\ \hline
      GreatWall & eth0 & 192.168.100.1/24 & eth1 & 192.168.200.1/24 \\ 
    \hline
  \end{tabular}
  \end{center}
  \caption{Router config}
  \label{table:complex-router}
  \end{table}
\end{center}

\begin{center}
  \begin{table}[htb]
  \begin{center}
  \begin{tabular}{| l | l | l | l | l | l | l |}
    \hline
      Hostname & Interface & MAC & Interface & MAC & Interface & MAC \\ \hline
      Switch & eth0 & 00:bb:cc:00:01:01 & eth1 & 00:bb:cc:00:02:02 & eth2 & 00:bb:cc:00:01:02 \\
    \hline
  \end{tabular}
  \end{center}
  \caption{Switch config}
  \label{table:complex-switch}
  \end{table}
\end{center}

After the devices are started we can use the bridges to sniff the packets at three points of the topology.
This will allow us to see how the network behaves. The following actions should be observed:
\begin{itemize}
	\item ARP request is made for the DNS server address
	\item Switches broadcasts the packet
	\item ARP response from the DNS server
	\item DNS request from the web browser
	\item DNS response from the DNS server
	\item TCP connection
	\item Web page request
	\item Web page response
	\item TCP hangup
\end{itemize}

For example to see the DNS request and response the following commands should be issued:
\lstset{language=zsh, caption=DNS request and respons, label=lst:dnsrr}
\begin{lstlisting}
tcpdump -ne -i tap0 "udp and dst port 53"
11:56:16.048401 b6:ff:df:62:6a:c2 > ce:8f:19:1a:81:d2, ethertype IPv4 (0x0800), length 70: 192.168.100.123.39465 > 192.168.100.200.53: 13412+ A? lklnet.org. (28)
tcpdump -ne -i tap0 "udp and src port 53"
11:55:28.439287 ce:8f:19:1a:81:d2 > b6:ff:df:62:6a:c2, ethertype IPv4 (0x0800), length 119: 192.168.100.200.53 > 192.168.100.123.44508: 42496*- 1/1/1 A 192.168.249.129 (77)
\end{lstlisting}

\fig[scale=0.5]{src/img/web.png}{img:complex-test}{Complex test topology}

After the normal test was run and the web page was displayed in the browser, we wanted to protect the DNS server from 
an outside DNS attack. For this we add the following rule to the router, which will now act as a firewall as well:
\lstset{language=zsh, caption=, label=complex-firewall}
\begin{lstlisting}
filter -A forward -d 192.168.100.200 -j DROP
\end{lstlisting}
An input interface might be specified to further limit the scope of rule, but it was not necessary in this case.
After the rule is added, the firewall blocks incoming packets for the DNS server.
