\chapter{Introduction}
\label{chapter:intro}

\section{Project Description}
\label{sec:proj}

Network simulators predict the behavior of a network, without actually needing any network devices.
This greatly reduces the time and cost required by setting up a network using real devices. Although
they have some disadvantages over real devices, speed and capabilities being the main ones, they
make it easier for people such researchers or students to test a diverse number of topologies under
different circumstances. They are also useful for engineers who do quality assurence and performance
testing.

Network simulators usually offer users the possibilty of creating network topologies, specifing the
network nodes and the links between those nodes. Apart from simulating the topology, some network
simulators have more advanced features such as packet playback or packet introspection.
There are also network simulators that are capable of describing not only networks, but also protocols
using domanin specific languages.

\textbf{\project} builds on the capabilities offered by the \textbf{Linux Kernel Library} to create
a complex network simulator capable of simulating end-to-end a complex network and a large number
of devices. This mean that each device will act like a linux box, offering a full TCP/IP stack.
More than this, the stack is proven to be correct by the users and developers of Linux. Advanced
users are not limited to simulating devices and protocols, they can also create new protocols by
adding them to the linux kernel. This method has some advantages over domain specific languages such
as the posibilty of using a more traditional language, C, or the possibilty of rapidly integrating
it back into the main Linux kernel. The disadvantage is that it is harder to write a protocol
in the linux kernel than writing it standalone.

%\subsection{Project Scope}
%\label{sub-sec:proj-scope}

\subsection{Project Objectives}
\label{sub-sec:proj-objectives}

educational

research

The system shouldn't be isolated from the outside world. It would be advantageous to be able
to include applications outside the LKL ecosystem.

\section{Related Work}
\label{sec:proj-related}

\subsection{Dynamips}

Dynamips started initialy as an Cisco 7200 emulator, but it now supports many other
Cisco platforms (3600, 3700 and 2600). It is an emulator, meaning that the emulated
devices act like the real devices, but also that is a little bit slow. It's
homepage \footnote{\url{http://www.ipflow.utc.fr/index.php/Cisco_7200_Simulator}}
states that it can go up to 1kpps. 

The emulated devices:
\begin{itemize}
  \item MIPS64 and PowerPC CPUs
  \item DRAM and SRAM
  \item Network modules:
    \begin{itemize}
      \item NM-1E
      \item NM-4E
      \item NM-1ESW
    \end{itemize}
  \item Temperature and Voltage sensors
\end{itemize}

Although the devices in dynamips work exactly like a real Cisco device, there are a few disadvanteges:
\begin{itemize}
  \item It requires an IOS image.
  \item High memory and CPU usage, thus limiting the number of devices that can be emulated on one computer.
  \item It can only emulate routers, not other devices.
  \item It can only simulate Cisco devices.
  \item The device code cannot be modified.
\end{itemize}

\subsection{Packet Tracer}

Packet tracer is a proprietary tool developed by Cisco. It is primary used as a learning tool
in Cisco Certified Network Assistent courses offered by Cisco.
Packet tracer is only a simulator and does not run code that runs on the Cisco devices, so
it is unsuitable for a realistic simulation of complex networks. It is also closed source, so
the devices cannot be modified.


\subsection{NS-2}
Ns(network simulator) is a discrete event network simulator used for networking research that is
mainly used for research. It uses a domain specific language to create network nodes, links and
hosts to generate traffic:
\lstset{language=zsh, caption=NS-2 topology, label=lst:ns2}
\begin{lstlisting}
set n0 [$ns node]
set n1 [$ns node
$ns duplex-link $n0 $n1 1Mb 10ms DropTail
set udp0 [new Agent/UDP]
$ns attach-agent $n0 $udp0
set cbr0 [new Application/Traffic/CBR]
$cbr0 set packetSize_ 500
$cbr0 set interval_ 0.005
$cbr0 attach-agent $udp0
set null0 [new Agent/Null] 
$ns attach-agent $n1 $null0
$ns connect $udp0 $null0
$ns at 0.5 "$cbr0 start"
$ns at 4.5 "$cbr0 stop"
\end{lstlisting}
The code in listing \ref{lst:ns2} tells ns-2 to create two network nodes connected by a link
and to attach a host and a traffic generate. The last two lines start to simulation.

Although it is a very powerfull tool, ns-2 has a very high learning curve, as it can be seen
in listing \ref{lst:ns2} and it does not run real devices.
