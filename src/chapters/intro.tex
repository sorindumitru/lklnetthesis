\chapter{Introduction}
\label{chapter:intro}

\section{Project Description}
\label{sec:proj}

\project

\subsection{Project Scope}
\label{sub-sec:proj-scope}

educational

research

\subsection{Project Objectives}
\label{sub-sec:proj-objectives}

The system shouldn't be isolated from the outside world. It would be advantageous to be able
to include applications outside the LKL ecosystem;

\subsection{Related Work}
\label{sub-sec:proj-related}

\subsubsection{Dynamips}

Dynamips started initialy as an Cisco 7200 emulator, but it now supports many other
Cisco platforms (3600, 3700 and 2600). It is an emulator, meaning that the emulated
devices act like the real devices, but also that is a little bit slow. It's
homepage \footnote{\url{http://www.ipflow.utc.fr/index.php/Cisco_7200_Simulator}}
states that it can go up to 1kpps. 

The emulated devices:
\begin{itemize}
  \item MIPS64 and PowerPC CPUs
  \item DRAM and SRAM
  \item Network modules:
    \begin{itemize}
      \item NM-1E
      \item NM-4E
      \item NM-1ESW
    \end{itemize}
  \item Temperature and Voltage sensors
\end{itemize}

Although the devices in dynamips work exactly like a real Cisco device, there are a few disadvanteges:
\begin{itemize}
  \item It requires an IOS image.
  \item High memory and CPU usage, thus limiting the number of devices that can be emulated on one computer.
  \item It can only emulate routers, not other devices.
  \item It can only simulate Cisco devices.
  \item The device code cannot be modified.
\end{itemize}

\subsubsection{Packet Tracer}

Packet tracer is a proprietary tool developed by Cisco. It is primary used as a learning tool
in Cisco Certified Network Assistent courses offered by Cisco.
Packet tracer is only a simulator and does not run code that runs on the Cisco devices, so
it is unsuitable for a realistic simulation of complex networks. It is also closed source, so
the devices cannot be modified.


\subsubsection{NS-2}
Ns(network simulator) is a discrete event network simulator used for networking research.
\todo{Learn more about NS-2}
