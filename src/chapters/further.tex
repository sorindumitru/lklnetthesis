\chapter{Further Work}
\label{chapter:further}

\subsubsection{Migration to other operating systems}

The current implementation of \textbf{\project} is a Linux only program. It would be a good
decision to port it to other operating systems, as it could gather more attention. 

Most of the code from \project is platform independent:
\begin{itemize}
  \item flex is available for most operating systems
  \item libreadline is available for most operating systems
  \item GTK is crossplatform
  \item There are implementations of POSIX on other platforms, and they are mostly POSIX-compliant
\end{itemize}

There are \textbf{LKL} environments for most operating systems(excluding MacOs) so it will be easily possible to port it
to other operating systems. The only part that has to be moved to a different operating system is the network driver.
But this can easily be done.

\subsubsection{Physical layer simulation}

The initial role of the hub was to offer a way to control the traffic. In the current implementation it is
only possible to view the packets that pass through the hub, but it would be beneficial to be able to also control
it.

Using the hub, it could be possible to simulate different Ethernet speeds by limiting the number(or size) of the packets
that pass through it.

\subsubsection{Scenario replay}

Another use for the hub could be to store the packets that pass through it. This way they can be later played back in the
order in which they appeared, allowing us to have a better look at what happens in the devices. This would be beneficial
in a learning environment.

\subsubsection{Distributed hypervisors}

The current implementation of the hypervisor is of only one hypervisor per topology. The number of devices that can be
simulated can be increased by having more hypervisors on different computers communicating between each other. Each hypervisor 
would manage its devices and communicate with the other hypervisors when they needed information about foreign devices.

\subsubsection{Port Zebra to LKL-net}

Gnu Zebra is a routing software package that provides TCP/IP based routing services with routing protocols support such
as RIP, OSPF or BGP. 

By porting Zebra to \textbf{\project} we would have a better way to manage the control pane of the router.
